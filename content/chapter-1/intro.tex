\chapter{\textsc{Общие принципы работы GPS}}

Глобальная навигационная спутниковая система (ГНСС) -- это сложная глобальная сеть, состоящая из трёх сегментов: космического сегмента (созвездия спутников), сегмента управления (наземных станций слежения) и пользовательского сегмента (приёмного оборудования).

Cистема работает с использованием созвездия спутников.  
Эти спутники вращаются в такой конфигурации, что в любой момент времени в любой точке Земли их видимое число составляет от четырех до десяти.
Каждый спутник непрерывно передаёт закодированный сигнал, содержащий информацию, которая однозначно определяет его самого и месторасположение в пространстве. 
Приёмник на поверхности Земли может выбрать один из этих сигналов и использовать его для определения расстояния до соответствующего спутника.  
Поскольку электромагнитный сигнал через атмосферу Земли распространяется примерно со скоростью света в вакууме $c$, то это расстояние можно рассчитать путём простого умножения времени между излучением и приёмом и $c$.

Используя несколько сигналов одновременно, приёмник может определить своё месторасположение на поверхности Земли методом трилатерации.
В теории, необходимо только три спутника, чтобы определить позицию приёмника, т.к. имеются только три неизвестных координаты: $x$, $y$ и $z$.   
Однако часы в приёмнике, которые используются для определения времени приёма сигнала, не синхронизованы с часами спутника.
Поскольку расчёт расстояния зависит от времени распространения сигнала, то любая ошибка во времени приёма переводится в ошибку расстояния.  
Эта ошибка часов приёмника является четвёртой неизвестной в системе уравнений, поэтому, на самом деле, необходимо четыре спутника для нахождения решения.

В настоящее время существуют две полностью функционирующих ГНСС: GPS (США) с июля 1995 года и ГЛОНАСС (Россия) с октября 2011 года, а также две находятся на стадии разработки: Galileo (ЕС) и BeiDou (Китай). 
Помимо этого, существуют региональные навигационные спутниковые системы, использование которых возможно только в определённых районах. 
К ним относятся QZSS (Япония) и IRNSS (Индия).  

Между каждой навигационной спутниковой системой имеются различия, но фундаментальные принципы остаются одинаковыми и применимыми к любой из упомянутых систем.
Основной фокус данной работы направлен на GPS.