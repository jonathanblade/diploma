\section{\textsc{Архитектура GPS}}

Архитектура GPS состоит из трёх основных сегментов: космического сегмента, сегмента управления и пользовательского сегмента.

\subsection*{\textbf{Космический сегмент}}

Космический сегмент GPS состоит из созвездия спутников, которые вращаются вокруг Земли в определённой конфигурации.
Эти спутники расположены так, что в любой момент времени в любой точки Земли их видимое число составляет от четырех до десяти, что гарантирует приёмнику стабильное определение своего месторасположения.  
Все спутники вращаются по почти круговым (эксцентриситет 0,02) средним околоземным орбитам (Medium Earth Orbit, MEO) на высоте примерно 20200 км. 
Орбитальный период равен половине сидерического дня, что соответствует примерно 11 часам и 58 минутам.  
Постепенно старые спутники выводятся и заменяется новыми спутниками, которые имеют более продолжительное время жизни и улучшенные технические характеристики. 
В настоящее время созвездие GPS включает в себя 32 действующих спутника, 31 из которых используется по целевому назначению и 1 выведен на техобслуживание. 

\subsection*{\textbf{Сегмент управления}}

Сегмент управления GPS представляет собой глобально распределённую сеть наземных станций, которые отслеживают и контролируют спутниковое созвездие.
Сегмент управления состоит из трёх элементов: станций мониторинга, наземных антенных станций и главной станции управления (Master Control Station, MSC).

В настоящее время сегмент управления включает в себя 16 станций мониторинга, расположенных в нескольких точках по всему миру \cite{GPS}.
Эти станции оснащены двухчастотным приёмником, дроссельной кольцевой антенной, которая устраняет эффект многолучёвости, двумя цезиевыми часами и метеорологическим оборудованием.      
Каждая станция собирает измерения псевдоадльности и фазы несущей от видимых спутников.   
Далее приёмник декодирует сигнал, чтобы получить навигационное сообщение и отправить всю эту информацию на MSC для обработки. 
Дополнительно каждая станция использует датчики для измерения давления, температуры и влажности, которые используются для моделей тропосферы.
Метеорологические данные также отсылаются на MSC для обработки.  

На момент написания работы, сегмент управления насчитывает 11 наземных антенных станций \cite{GPS}.
Эти антенны S-диапазона обеспечивают слежение, телеметрию и контроль спутников на основе команд от MSC. 
Наземные антенны устанавливают сеансы управления только с одним спутником, в то время как MSC способна координировать сеансы управления одновременно для нескольких спутников путём использования различных наземных антенн.
Помимо этого, наземные антенные станции отвечают за передачу навигационного сообщения, генерируемого MSC.

MSC находится на авиабазе Шривер (Колорадо) под управлением ВВС США. 
Резервная MSC находится в городе Гейтерсберг (Мэриленд).
Основная функция MSC заключается в создании и передаче навигационного сообщения спутниковому созвездию.
Также MSC отвечает за ряд других задач: мониторинг месторасположения и состояния спутников, расчёт предполагаемых параметров часов и координат спутников, синхронизация GPST (GTS Time) и UTC, контроль целостности навигационного сообщения, проверка и регистрация навигационных данных, передаваемых в пользовательский сегмент, а также коррекция положения спутников для устранения орбитальных аномалий или перестройки созвездия в случае отказа одного из спутников. 

\subsection*{\textbf{Пользовательский сегмент}}

К пользовательскому сегменту GPS относятся все приёмники, способные принимать и декодировать сигналы спутников GPS.
В настоящее время наиболее распространены два типа приёмников: одночастотные (бюджетные и неточные) и двухчастотные (более дорогие и точные).