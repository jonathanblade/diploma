\section{\textsc{Сигналы GPS}}

Сигналы GPS состоят из трёх основных компонентов: несущей сигнала, кода ранжирования и навигационного сообщения.

\subsection*{\textbf{Несущая сигнала}}

Несущая сигнала представляет собой синусоидальную электромагнитную волну, которая колеблется на определённой частоте (частоте несущей). 
При своём распространении несущая подвергается различным частотно-зависимым эффектам, поэтому выбор частоты несущей играет принципиальную роль.  

Средой распрострарения луча приёмник-спутник является атмосфера.
Атмосфера может быть разделена на две области, которые оказывают значимое влияние на распространение электромагнитных волн.
Эти области называются ионосферой и тропосферой.
Основные ионосферные эффекты, такие как поглощение, групповая задержка, рефракция и другие, в первом приближении пропорциональны $\frac{1}{f^2}$, где $f$ -- частота несущей. 
Тропосферные же эффекты наоборот -- становятся более выраженными при высоких частотах.
Таким образом, частота несущей должна находиться в определённом диапазоне: примерно от 10 МГц (ниже сигнал не проходит ионосферу) до 5 ГГц (выше сигнал значительно затухает в тропосфере).
Это обуславливает тот факт, что сегмент управления для слеживания, телеметрии и контроля спутников использует S-диапазон (2-4 ГГц), а спутники передают свои сигналы в L-диапазоне (1-2 ГГц).

В настоящее время GPS использует три отдельных несущих, которые получаются от одного атомного тактового генератора с частотой $f_0=10,23$ МГц:
\begin{equation}
\label{eq-carriers}
\begin{aligned}
&L1=154\times f_0=1575,42\,\text{МГц}\Rightarrow\lambda_{L1}=19,0\,\text{см} \\
&L2=120\times f_0=1227,60\,\text{МГц}\Rightarrow\lambda_{L2}=24,4\,\text{см} \\
&L5=115\times f_0=1176,45\,\text{МГц}\Rightarrow\lambda_{L5}=25,5\,\text{см} 
\end{aligned}
\end{equation}
Несущие L1 и L2 передаются всеми поколениями спутников GPS, а L5 -- только спутниками Блока IIF и новее. 

\subsection*{\textbf{Код ранжирования}}

Код ранжирования также известный, как PseudoRaundom Noise (PRN) код, модулирует несущие сигналов GPS.
PRN код представляет собой детерминированную битовую последовательность, доступ к которой имеют только приёмники GPS.
Любые другие приёмники интерпретируют этот код, как случайный шум, поэтому игнорируют его.

PRN код уникален (ортогонален) для каждого спутника и бывает нескольких типов.
Три изначальных кода -- L1 C/A, L1 P(Y) и L2 P(Y) -- передаются всеми спутниками с момента запуска GPS. 
Непрерывная модернизация GPS привела к появлению дополнительных PRN кодов, включая L1C, L2C, а также M кода для L1 и L2.
В табл. \ref{tab-prn} собрана информация о некоторых важных параметрах каждого существующего PRN кода.
\begin{table}[h]
\caption{\centerline{Характеристики текущих и будущих сигналов GPS \cite{Subirana2013}.}}
\label{tab-prn}
\resizebox{\textwidth}{!}{
\begin{tabular}{|c|c|c|c|c|c|}
\hline
Сигнал              & Частота несущей, МГц     & PRN код & Метод модуляции           & Частота кода, МГц       & Доступ                       \\ \hline
\multirow{4}{*}{L1} & \multirow{4}{*}{1575,42} & C/A     & BPSK(1)                   & 1,023                   & \multirow{2}{*}{Гражданский} \\ \cline{3-5}
                    &                          & L1C     & MBOC(6,1,1/11)            & 1,023                   &                              \\ \cline{3-6} 
                    &                          & P(Y)    & BPSK(10)                  & 10,23                   & \multirow{2}{*}{Военный}     \\ \cline{3-5}
                    &                          & M       & BOC(10,5)                 & 5,115                   &                              \\ \hline
\multirow{4}{*}{L2} & \multirow{4}{*}{1227,60} & L2CM    & \multirow{2}{*}{BPSK(1)}  & \multirow{2}{*}{0,5115} & \multirow{2}{*}{Гражданский} \\ \cline{3-3}
                    &                          & L2CL    &                           &                         &                              \\ \cline{3-6} 
                    &                          & P(Y)    & BPSK(10)                  & 10,23                   & \multirow{2}{*}{Военный}     \\ \cline{3-5}
                    &                          & M       & BOC(10,5)                 & 5,115                   &                              \\ \hline
\multirow{2}{*}{L5} & \multirow{2}{*}{1176,45} & L5I     & \multirow{2}{*}{QPSK(10)} & \multirow{2}{*}{10,23}  & \multirow{2}{*}{Гражданский} \\ \cline{3-3}
                    &                          & L5Q     &                           &                         &                              \\ \hline
\end{tabular}}
\end{table}

\subsection*{\textbf{Навигационное сообщение}}

Помимо PRN кода, несущие сигналов GPS дополнительно модулируются навигационным сообщением с частотой 50 Гц, которое в себе содержит:
\begin{description}[wide]
\item \textit{Эфемериды.}
Орбитальные данные (элементы Кеплера и параметры возмущения), используемые для предсказания траектории спутника.
\item \textit{Информацию о спутниковых часах.}
Коэффициенты полинома 2-ой степени, которые соответствуют смещению, дрейфу и скорости дрейфа (старения) атомных часов на борту спутника.  
\item \textit{Информацию о состоянии спутника.}
Валидационные параметры, указывающие следует ли доверять сигналу спутника или нет.
\item \textit{Альманах.}
Эфемериды, информация о часах и состоянии всего спутникового созвездия, но с пониженной точностью. 
\item \textit{Дополнительные параметры.}
Параметры ионосферной коррекции для одночастотных приёмников, данные для преобразования между GPST и UTC. 
\end{description}