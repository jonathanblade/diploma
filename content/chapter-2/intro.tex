\chapter{\textsc{Методы учёта влияния ионосферы в задачах прицезионного GPS позиционирования}}

Precise Point Positioning (PPP) -- это метод абсолютного позиционирования, который использует кодовые или фазовые измерения сигнала в комбинации с точными эфемеридами спутников \cite{Zumberge1997}. 
С момента полного отключения режима SA\footnote{Режим искусственного снижения точности GPS. Полностью отключен правительством США с 1 мая 2000 года.} (Selective Availability), основным источником ошибки PPP является задержка сигнала в ионосфере. 
В настоящее время существуют несколько методов коррекции ионосферной задержки, которые могут применяться как по отдельности, так и вместе.
В связи с тем, что ионосфера является дисперсионной средой, то величина ионосферной задержки зависит от частоты сигнала.
Поэтому пользователи двухчастотных приёмников практически полностью (до 99,9\%, что соответствует вкладу ионосферной ошибки 1-го порядка) могут устранить ионосферную ошибку путём комбинирования кодовых или фазовых измерений сигнала на разных частотах.
Пользователи же более дешёвых одночастотных приёмников должны применять другие методы.
Например, использовать ионосферные модели или применять разностные методы. 