\section{\textsc{Ионосфера и её влияние на сигналы GPS}}
\label{chap-2-1}

Ионосфера -- это область верхней атмосферы Земли на высотах примерно от 50 км до 1000 км и более, в которой количество свободных электронов достаточно, чтобы оказывать влияние на распространение электромагнитных волн, в частности сигналов GPS.

Главным механизмом образования свободных электронов в ионосфере является ионизация.
Ионизация -- это процесс, при котором электрически нейтральный атом или молекула теряет либо приобретает электрон, что приводит к образованию положительного или отрицательного иона, соответственно.
В верхних слоях атмосферы высокоэнергетичные солнечные фотоны сталкиваются с частицами газа, лишают их электронов и ионизируют их.
Фотоионизация или ионизация под действием электромагнитных волн может быть математически представлена уравнением:
\begin{equation}
a+\frac{hc}{\lambda}\rightarrow a^{+}+e^{-}    
\end{equation}
где
$a$ -- произвольная нейтральная частица;
$\frac{hc}{\lambda}$ -- энергия падающего фотона с длиной волны $\lambda$;
$a^{+}$ -- положительный ион;
$e^{-}$ -- электрон.
Этот процесс происходит только в том случае, если энергия падающего фотона превышает энергию ионизации частицы.
Точка, когда энергия фотона равна энергии ионизации называется пределом ионизации.
Поскольку энергия фотона обратно пропорциональна его длине волны, предел ионизации может быть выражен в единицах измерения длин волн.
Если длина волны падающего фотона меньше предела ионизации, то атом или молекула будут ионизованы.
Пределы ионизации нескольких химических веществ, распространённых в ионосфере, приведены в табл. \ref{tab-ion-limits}.  
\begin{table}[h]
\caption{\centerline{Пределы ионизации химических веществ,}
\centerline{присутствующих в ионисфере \cite{Rishbeth2003}.}}
\label{tab-ion-limits}
\centering
\begin{tabular}{|c|c|}
\hline
Элемент                & Предел ионизации, нм  \\ \hline
$\text{N}_2$           & 79,6                  \\ \hline
$\text{O}$             & 91,1                  \\ \hline
$\text{H}$             & 91,1                  \\ \hline                
$\text{O}_2$           & 102,7                 \\ \hline
$\text{NO}$            & 134                   \\ \hline
\end{tabular}
\end{table}

Безусловно, наиболее распространёнными химическими веществами в атмосфере являются молекулярный азот ($\text{N}_2$) и молекулярный кислород ($\text{O}_2$).
Эти молекулы могу быть ионизованы посредством следующих реакций: 
\begin{equation}
\label{eq-react1}
\begin{aligned}
&\text{N}_2+\frac{hc}{\lambda}\rightarrow\text{N}_2^{+}+e^{-} \\
&\text{O}_2+\frac{hc}{\lambda}\rightarrow\text{O}_2^{+}+e^{-}     
\end{aligned}
\end{equation}
Образовавшиеся положительные ионы могут диссоциативно рекомбинировать со свободными электронами.
Это приводит к двум нейтральным атомам, как показано в следующих реакциях:
\begin{equation}
\label{eq-react2}
\begin{aligned}
&\text{N}_2^{+}+e^{-}\rightarrow\text{N}+\text{N} \\
&\text{O}_2^{+}+e^{-}\rightarrow\text{O}+\text{O} 
\end{aligned}
\end{equation}
Атомный кислород может быть ионизирован сам, либо может участвовать в реакции замещения с $\text{N}_2^{+}$: 
\begin{equation}
\begin{aligned}
\label{eq-react3}
\text{O}+\frac{hc}{\lambda}&\rightarrow\text{O}^{+}+e^{-} \\
\text{N}_2^{+}+\text{O}&\rightarrow\text{NO}^{+}+\text{N}  
\end{aligned}
\end{equation}
Ионы атомарного кислорода могут принимать участие в реакциях замещения с $\text{N}_2$ или $\text{O}_2$:
\begin{equation}
\label{eq-react4}
\begin{aligned}
&\text{O}^{+}+\text{N}_2\rightarrow\text{NO}^{+}+\text{N} \\
&\text{O}^{+}+\text{O}_2\rightarrow\text{O}_2^{+}+\text{O} 
\end{aligned}
\end{equation}
И наконец, ионы оксида азота могут диссоциативно рекомбинировать:
\begin{equation}
\label{eq-react5}
\text{NO}^{+}+e^{-}\rightarrow\text{N}+\text{O}
\end{equation}
Ряд физических и химических процессов информирует об относительном количестве каждого вида иона на определённой высоте.
Хотя атмосфера богата $\text{N}_2$, ионы $\text{N}_2^{+}$ не существуют в больших количествах, т.к. они очень быстро реагируют с помощью уравнений \eqref{eq-react2} и \eqref{eq-react3}. 
На малых высотах, где молекулярная плотность высока, преобладают молекулы $\text{N}_2$ и $\text{O}_2$.
Ионы $\text{O}^{+}$ быстро реагируют с этими молекулами согласно уравнениям \eqref{eq-react4}.
Прямая ионизация NO образует $\text{NO}^{+}$, но атмосфера не богата на NO.
Несмотря на это, $\text{NO}^{+}$ существует в изобилии на малых высотах из-за реакций \eqref{eq-react3} и \eqref{eq-react4}. 
На больших высотах $\text{N}_2$ и $\text{O}_2$ встречаются реже, поэтому реакции \eqref{eq-react4} менее вероятны.
Когда они происходят, образующиеся ионы $\text{NO}^{+}$ и $\text{O}_2^{+}$ быстро рекомбинируют с большим числом свободных электронов посредством реакций \eqref{eq-react2} и \eqref{eq-react5}.
Молекулярные ионы могут быстро и легко рекомбинировать с электронами посредством диссоциативной рекомбинации, но этот механизм не доступен для атомарных ионов.
Поэтому атомарные ионы могут излучать энергию в виде света.
Этот процесс называется излучательной рекомбинацией.
Излучательная рекомбинация неэффективна и происходит гораздо реже, чем диссоциативная рекомбинация.
Также ионы могу рекомбинировать через столкновения.
Согласно законам сохранения импульса и энергии, а также принципам квантовой механики, для рекомбинации в столкновении должны участвовать три атомарных иона и только два молекулярных иона \cite{Rishbeth2003}.
На малых высотах столкновения как двух, так и трёх ионов происходят с некоторой регулярностью из-за высокой плотности атмосферы.
На больших высотах столкновения трёх ионов становятся гораздо реже.
Поэтому $\text{O}^{+}$ будет менее часто рекомбинировать, чем $\text{N}_2^{+}$, $\text{O}_2^{+}$ и $\text{NO}^{+}$.   
Таким образом, на малых высотах наиболее распространены ионы $\text{O}_2^{+}$ и $\text{NO}^{+}$, а также присутствуют $\text{N}_2^{+}$.
Эти молекулярные ионы доминируют примерно до 200 км, после чего доминирующим ионом становится $\text{O}^{+}$ \cite{Rishbeth2003}.

Обычно ионосфера делится на три слоя, которые называются D, E и F.
Эти обозначения были введены Э.В. Эплтоном в 1920-х годах.
Эплтон был первым, кто экспериментально подтвердил существование ионосферы, продемонстрировав, что радиоволны могу отражаться от верхней атмосферы.
Первый обнаруженный слой Эплтон назвал буквой E, чтобы подчеркнуть тот факт, что электрические поля были отражены.
Впоследствии он обнаружил ещё два дополнительных слоя: один выше слоя E, который обозначил буквой F, а другой ниже слоя E, который обозначил буквой D. 
С тех пор было обнаружено, что слой F может делиться на ещё две области, которые имеют разные свойства: F1 и F2.
Следуя тому же принципу, слой E иногда делится на области E1 и E2, хотя это редко встречается в литературе.
Самую низкую часть ионосферы иногда называют слоем C, но этот слой не важен для большинства приложений, поэтому он либо включается в слой D, либо полностью игнорируется.
Концентрация свободных электронов непрерывно изменяется с высотой, поэтому определение слоёв является произвольным.
Их границы определяются прежде всего различиями в химическом составе и физических процессах ионосферы.
Эти факторы очень динамичны и значительно зависят от широты, времени суток, времени года и цикла солнечной активности.
Таким образом, границы каждого слоя могут быть определены только приблизительно и варьируются в пределах 10-20 км.

Слой C -- самый нижний слой ионосферы, расположенный на высотах примерно 50-70 км над поверхностью Земли.
На эту глубину способны проникать только фотоны очень высоких энергий (космические гамма-лучи с энергией МэВ и более).
Эти фотоны способны ионизировать все химические вещества в атмосфере.
Солнце лишь изредка излучает гамма-лучи, например, во время солнечных вспышек, которые вызывают внезапные ионосферные возмущения (Sudden Ionospheric Disturbances, SID).
SID обычно длятся около получаса \cite{Rishbeth2003}, но тем не менее могу вызывать проблемы для распространения радиоволн.    
Обычно гамма-лучи испускаются от далёких чёрных дыр, квазаров, сверхновых и нейтронных звёзд.
Тем не менее их поток довольно мал\footnote{Для сравнения: общий солнечный поток на всех длинах волн в верхней части атмосферы составляет $\mathcal{O}(10^{6})\,\text{эрг}\,\text{см}^2\,\text{с}^{-1}$.} и составляет $\mathcal{O}(10^{-12})\,\text{эрг}\,\text{см}^2\,\text{с}^{-1}$ ($1\,\text{эрг}\,\text{с}^{-1}=10^{-7}\,\text{Вт}$).

Cлой D -- область ионосферы на высотах примерно 70-90 км над поверхностью Земли.
На эту глубину могут проникать фотоны с длиной волны 1 нм и меньше. 
Это означает, что помимо гамма-лучей жёсткое рентгеновское излучение ($\lambda=0,1\div 1$ нм), способное ионизировать $\text{N}_2$ и $\text{O}_2$, также достигает слоя D.
Поток солнечного рентгеновского излучения зависит от солнечной активности, которая включает в себя 11-летний цикл, а также солнечные вспышки.
Поскольку в течение солнечного цикла поток рентгеновского излучения в большинстве случаев мал, первичный источник ионизации в слое D, на самом деле, обусловлен спектральной линией водорода, проходящей через окно сечения поглощения $\text{O}_2$ \cite{Rishbeth2003}.
Эта спектральная линия (Лайман-$\alpha$) имеет длину волны 126,6 нм и способна ионизировать оксид азота (NO) в атмосфере. 
Хотя NO относительно менее распространён, чем $\text{N}_2$ и $\text{O}_2$, большой поток Лайман-$\alpha$ ($5\,\text{эрг}\,\text{см}^2\,\text{с}^{-1}$) способен продуцировать значительное количество ионов $\text{NO}^{+}$.
На высотах слоя D наиболее распространёнными ионами являются $\text{NO}^{+}$ и $\text{O}_2^{+}$, в то время как $\text{O}^{+}$ и $\text{N}_2^{+}$ существуют в меньших количествах.
Водяной пар присутствует на высотах ниже примерно 85 км и может взаимодействовать с $\text{NO}^{+}$, образуя более крупные молекулярные ионы, такие как $\text{H}_3\text{O}^{+}$, $\text{H}_5\text{O}_2^{+}$ и $\text{H}_7\text{O}_4^{+}$. 
Процессы рекомбинации и присоединения играют важную роль для слоя D.
Высокая плотность приводит к большому количеству столкновений.
Ночью электроны могу присоединиться к нейтральным молекулам с образованием отрицательных ионов, которые впоследствии могут рекомбинировать с положительными ионами.
По этой причине электронные потери в D слое относительно велики.
Тем не менее в течение дня эти потери в некоторой степени компенсируются.
Как результат, слой D имеет более низкую электронную концентрацию, чем слои E и F, которая составляет примерно $10^8\div10^9\,\text{м}^{-3}$ в зависимости от солнечной активности. 
Ночью при отсутствии солнечного излучения процессы рекомбинации и присоединения приводят к тому, что слой D почти полностью исчезает, и лишь небольшая часть ионизации остаётся из-за космических лучей.

Cлой E простирается на высотах примерно 90-140 км над поверхностью Земли.
Мягкое рентгеновское излучение ($\lambda=1\div10$ нм) и экстремальное ультрафиолетовое (Extreme UltraViolet, EUV) излучение с $\lambda=91,2\div102,6$ нм постоянно проникают на эту глубину и вызывают ионизацию.
Стоит отметить, что излучения с $\lambda=10\div91,1$ нм также теоретически способны проникать на эту глубину, но на практике они сильно поглощаются слоями F1 и F2. 
В спектр EUV включена линия излучения Лайман-$\beta$ с $\lambda=102,6$ нм, способная ионизировать $\text{O}_2$.
Ионы $\text{NO}^{+}$ и $\text{O}_2^{+}$ являются наиболее распространёнными, тогда как $\text{O}^{+}$ и $\text{N}_2^{+}$ имеют вторичное содержание.
Электронная концентрация слоя E больше, чем слоя D, и составляет порядка $10^{11}\,\text{м}^{-3}$.
Ночью ионизация значительно уменьшается из-за рекомбинации до примерно $5\times10^{9}\,\text{м}^{-3}$.
Иногда на высотах 100-120 км можно обнаружить плотный участок ионизации.
Этот область называется спорадическим слоем E и обозначается, как Es.
Образование слоя Es никак не связано с образованием стандартного слоя E и не следует какой-либо переодической структуре.

Слой F1 находится на высотах примерно 140-210 км над поверхностью Земли.
Ионизация в этой области обусловлена EUV излучением с $\lambda=10\div91,1$ нм.
Как и в слое E, здесь преобладают ионы $\text{O}_2^{+}$ и $\text{NO}^{+}$ и вторично распространены $\text{O}^{+}$ и $\text{N}_2^{+}$.
Электронная концентрация в слое F1 больше, чем в слоях D и E, и в зависимости от солнечной активности составляет порядка $10^{11}\div10^{12}\,\text{м}^{-3}$.  
Ночью ионизация в этой области исчезает не полностью.
Вместо этого слой F1 объединяется со слоем F2 и образует единую область -- слой F.
Слой F2 простирается на высотах примерно 210-600 км над поверхностью Земли.
Здесь доминирует ионы $\text{O}^{+}$.
Слой F2 имеет самую высокую электронную концентрацию, которая достигает примерно $8\times10^{12}\,\text{м}^{-3}$.
Помимо этого, он остаётся сильно ионизованным в ночное время из-за транспортных эффектов, которые вызывают нисходящий вертикальный дрейф ионизации от протоносферы.

Значительно выше максимума слоя F2 (на высотах более 700 км) количество ионов $\text{O}^{+}$ начинает уменьшаться.
На их место приходят ионы $\text{H}^{+}$ и $\text{He}^{+}$, причём $\text{H}^{+}$ являются более распространёнными.
Ионы $\text{H}^{+}$ представляю собой просто свободные протоны, поэтому эта область называется протоносферой.
Электронная концентрация в этой области ниже, чем в слое F2, и составляет около  $10^{10}\,\text{м}^{-3}$.
Протосфера и слой F2 постоянно обмениваются ионами посредством реакции:
\begin{equation}
\text{O}^{+}+\text{H}\rightleftarrows\text{H}^{+}+\text{O}
\end{equation}
В течение дня восходящий диффузионный поток вызывает прямую реакцию, при которой ионизация переносится из слоя F2 в протоносферу.
Ночью имеет место обратная реакция, т.е. нисходящий диффузионный поток обеспечивает слой F2 ионизацией из протоносферы.

Электромагнитные волны, проходящие через ионосферу, подвержены ряду эффектов, многие их которых возникают непосредственно из-за взаимодействия со свободными электронами.
Эти эффекты зависят не только от частоты сигнала, но и от полного электронного содержания (Total Electron Content, TEC):
\begin{equation}
\text{TEC}=\int_S N_e(s)\,ds    
\end{equation}
где 
$N_e(s)$ -- функция концентрации свободных электронов.  
TEC является пространственно-временным параметром, зависящим от состояния ионосферы на пути сигнала межу спутником и приёмником и обычно измеряется в TECU (Total Electron Content Units), где $1\,\text{TECU}=10^{16}\,\text{м}^{-2}$.
Прил. \ref{ap-1} резюмирует наиболее важные эффекты, которые оказывает ионосфера на распространение радиоволн различных частот \cite{ITU-R2015}.
Условия, при которых эта таблица была получена, соответствуют наихудшему случаю для распространения волн в невозмущённой ионосфере, т.е. дневным часам в период высокой солнечной активности. 
Можно заметить, что для рабочих частот GPS значимыми эффектами являются групповая задержка сигнала и фарадеевское вращение.
Все остальные эффекты являются несущественными и в большинстве случаев ими можно пренебречь.