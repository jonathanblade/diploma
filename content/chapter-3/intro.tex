\chapter{\textsc{Исследование точности позиционирования GPS в условиях естественной ионосферной возмущённости}}

За последние несколько десятков лет было много работ, посвящённых влиянию геомагнитных бурь на точность позционирование GPS.
В настоящее время наиболее популярным методом относительного позиционирования является Real-Time Kinematic (RTK) метод.
Как правило, для базовых линий порядка 100 км RTK может достичь точности примерно на сантиметровом уровне \cite{Rizos2002}.
Однако геомагнитные бури оказывают существенное влияние на это \cite{Jacobsen2012}. 
Из-за быстрой декорреляции ионоферной ошибки во время сильной бури 29 октября 2003 года процент успешной разрешимости мгновенной неоднозначности RTK снизился с 94\% (спокойный день) до 31\%.  
В результате, для базовой линии равной 121 км ошибка позиционирования по вертикали превысила 0,5 м.
Подобные результаты работы \cite{Bergeot2011} также указывают, что в регионе Европы в период геомагнитной бури 30 октября 2003 года точность позиционирования GPS в кинематическом режиме достигла 12,8 см, 8,1 см и 26,1 см для горизонтальных и вертикальной компонент, соответственно, в то время как при спокойных ионосферных условиях было не более 2,5 см.
За исключением самих ионосферных возмущений, на точность позиционирования RTK также влияет ориентация базовой линии во время бури.
Максимальные и стандартные отклонения значений ошибок позиционирования для базовых линий с ориентацией север-юг больше, чем для базовых линий с ориентацией запад-восток \cite{Lejeune2012}. 
В более новой работе \cite{Jacobsen2016}, сфокусированной уже на регионе Норвегии, представлен подробный анализ эффективности сети RTK и фазовых мерцаний во время сильной геомагнитной бури в День святого Патрика (17 марта) 2015 года.   

Другим популярным методом позиционирование является PPP метод, исследование которого на влияние геомагнитных бурь получило лишь ограниченное внимание. 
Как упоминалось ранее, PPP --  это метод абсолютного позиционирования, который позволяет достичь точности на сантиметровом уровне при использовании двухчастотных измерений в комбинации с точными эфемеридами спутников \cite{Zumberge1997,Lou2016}. 
В работе \cite{Jacobsen2016} также было показано, что при одних и тех же условиях возмущения ионосферы точность PPP лучше, чем у RTK.
Однако там были рассмотрены только три станции в пределах 55-70\degree N.
Чтобы получить более репрезентативные результаты, авторы работы \cite{Luo2018} изучили несколько сильных геомагнитных бурь по данным примерно 500 станций IGS в период 24-го цикла солнечной активности, включая бурю 17 марта 2015 года.     
Их результаты указывают на увеличение ошибки позиционирования (PPP) до 0,32 м на средних широтах и более 1 м на высоких широтах, относительно 0,163 м в спокойный день.

В данной же работе используются общедоступные данные как международной (IGS), так и региональных сетей ГНСС. 
Таким образом, по общей совокупности задействованы порядка 4500 приёмных станций, расположенных по всему миру. 
Основной задачей является комплексный анализ точности позиционирования GPS в различных широтно-долготных регионах при использовании PPP в двухчастотном кинематическом режиме во время нескольких сильных гелиогеофизических явлений (солнечных вспышек \cite{Yasyukevich2018} и геомагнитных бурь) в период 24-го цикла солнечной активности. 