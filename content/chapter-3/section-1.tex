\section{\textsc{Исходные данные и метод оценки ошибки PPP}}

В работе используются двухчастотные измерения GPS примерно 4200 приёмных станций, расположенных по всему миру.
Это общедоступные данные, которые предоставляются сетями ГНСС, в список которых входят IGS \cite{Dow2009}, CHAIN \cite{Jayachandran2009}, TrigNet, GEOSCIENCE AUSTRALIA, CORS, LPIM, New Zealand (GEONET), Sonel, Unavco и KASI.

Чтобы оценить точность позиционирования, для каждой станции рассчитываются точные координаты (PPP) в кинематическом режиме при помощи программного обеспечения с открытым исходным кодом GAMP \cite{Zhou2018}. 
При решении навигационной задачи используется двухчастотная модель PPP, которая основана на недифференцированных и некомбинированных измерениях GPS.
Линеризованные уравнения исходных измерений псевдодальности \eqref{eq-pr} и фазы несущей \eqref{eq-cr2} имеют вид \cite{Zhou2018}:
\begin{equation}
\begin{aligned}
&R_r^s=\vec{u}\cdot\vec{x}+c(\delta_r-\delta^s)+F\times T_V+\alpha_fI_{L1}+c(d_r-d^s)+e_r^s \\
&\lambda\Phi_r^s=\vec{u}\cdot\vec{x}+c(\delta_r-\delta^s)+F\times T_V-\alpha_fI_{L1}+\lambda(N+b_r-b^s)+e_r^s
\end{aligned}
\end{equation}  
где 
$\vec{u}$ -- единичный вектор в направлении от приёмника к спутнику; 
$\vec{x}$ -- вектор приращений положения приёмника относительно априорной позиции.
Таким образом, в решении учитываются смещения часов приёмника $\delta_r$ и спутника $\delta^s$ (путём применения точных эфемерид от IGS), вертикальная влажная тропосферная задержка $T_V$ (напомним, что $F$ -- наклонный коэффициент, определяемый выражением \eqref{eq-slant-factor}), ионосферная задержка $\alpha_fI_{L1}$ ($\alpha_f=\frac{f_{L1}^2}{f^2}$), а также некалиброванные кодовые ($d$) и фазовые ($b$) задержки приёмника и спутника.   
Помимо этого, заранее моделируется смещение и вариации фазовых центров антенн приёмника и спутника, релятивистские эффекты, эффект Саньяка, гидростатическая задержка, приливные эффекты и wind-up эффект.

За истинные геоцентрические координаты cтанции принимаются её медианные значения, полученные из решения за 24 часа измерений без учёта времени сходимости решения.
Общая ошибка позиционирования считается, как разница между текущей ($x$, $y$, $z$) и истинной ($x_0$, $y_0$, $z_0$) позицией:
\begin{equation}
\sigma_{xyz}=\sqrt{(x-x_0)^2+(y-y_0)^2+(z-z_0)^2}
\label{eq-3d-error}
\end{equation}

Чтобы оценить динамику ошибки позиционирования для различных широтно-долготных регионов, рассчитываются среднезональные значения (т.е. ошибка усредняется по ячейкам размером 2,5\degree~по широте и доготе) для двух областей: Европа-Азия ($0\div180\degree \text{E}$) и Америка-Антлантика ($0\div180\degree \text{W}$).  