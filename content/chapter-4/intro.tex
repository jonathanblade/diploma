\chapter{\textsc{Исследование точности позиционирования GPS в условиях искусственной ионосферной возмущённости}}

Эксперименты по изменению состояния ионосферы путём её нагрева с помощью высокочастотных (ВЧ) волн ведутся уже довольно давно (примерно с 60-70-х годов прошлого века). 
Всего в мире для этих целей существовало более 10 экспериментальных установок, но к настоящему времени осталось функционировать только 4: на высокий широтах -- HARP (Аляска, США) и EISCAT/Heating (Тромсё, Норвегия); на низких широтах -- Arecibo (Аресибо, Пуэрто-Рико); на средних широтах -- СУРА (Нижний Новгород, Россия).
За это время в ряде работ \cite{Gurevich2007, Streltsov2018, Stubbe1996, Erukhimov1987, Leyser2001, Stubbe1997, Grach2016}, посвящённых исследованию основных механизмов генерации ионосферных возмущений около подобных нагревательных стендов, были идентифицированы ионосферные неоднородности размером от нескольких сантиметров до сотни километров.
Также были проведены исследования параметров искусственных перемещающихся ионосферных возмущений (ПИВ), но уже вдали (вплоть до 1000 км) от нагревательных стендов СУРА \cite{Chernogor2011, Chernogor2013} и HARP \cite{Mishin2012, Pradipta2015}.
А в работе \cite{Kunitsyn2012} с помощью радиотомографических методов впервые была восстановлена пространственная структура таких ПИВ.

Известно, что наибольший эффект на распространение электромагнитных волн в ионосфере оказывают неоднородности электронной концентрации размера порядка радиуса первой зоны Френеля.
Неоднородности такого масштаба являются основной причиной амплитудных мерцаний (быстрых и случайных флуктуаций амплитуды) навигационного сигнала.
Неоднородности же большего масштаба приводят к фазовым мерцаниям (соответственно, быстрым и случайным флуктуациям фазы сигнала).
Эти эффекты также были зарегистрированы во время экспериментов на стендах HARP \cite{Bernhardt2016}, EISCAT/Heating \cite{Tereshchenko2000} и СУРА \cite{Tereshchenko2004}. 
Таким образом, подобные искусственные воздействия оказывают влияние на параметры навигационного сигнала и потенциально могут приводить к дополнительной ионосферной задержке \cite{Jakowski1997}, следовательно, являться фактором ухудшения точности позиционирования ГНСС, в частности GPS.

Согласно работе \cite{Marques2018}, точность позиционирования PPP в двухчастотном кинематическом режиме во время мерцаний снижается в несколько раз и может достигать 1-2 м.
Также установлено, что интенсивность искусственных неоднородностей сравнима с интенсивностью неоднородностей во время сильных гелиогеофизических явлений \cite{Frolov2017}.
Как упоминалось ранее (см. ГЛАВА 3), в настоящее время имеются работы, в которых производилась оценка ошибки позиционирования во время солнечных вспышек \cite{Yasyukevich2018}, а также геомагнитных бурь \cite{Jacobsen2012, Bergeot2011, Lejeune2012, Jacobsen2016, Luo2018}.
Однако стоит отметить, что прямой анализ точности позиционирования GPS при искусственной ионосферной возмущённости ни в одной из работ не проводился.   
Таким образом, основной задачей является комплексный анализ точности позиционирования GPS при использовании PPP в двухчастотном кинематическом режиме во время нескольких рабочих сеансов радиокомплекса СУРА.