\chapter*{ЗАКЛЮЧЕНИЕ}
\addcontentsline{toc}{chapter}{ЗАКЛЮЧЕНИЕ} 

В работе проведено исследование точности позиционирования PPP в условиях естественной и искусственной ионосферной возмущённости. 
Подробно рассмотрены две геомагнитные бури за 25-26 августа 2018 года и 21-23 июня 2015 года. 
Получены и проанализированы глобальные распределения полной ошибки позиционирования, а также зависимости усреднённой ошибки от времени и широты для восточного и западного полушарий. 
Дополнительно проведено сравнение со спокойными днями. 
Во время главной фазы бури зафиксировано ухудшение точности позиционирования до 5 раз, которое пространственно коррелирует с областями максимальных вариаций TEC.  
Также рассмотрены эксперименты на нагревательном стенде СУРА за 23 августа 2010 года и 19-20 сентября 2016 года.
Построены динамики полной ошибки позиционирования для нескольких станций, расположенных на разном расстоянии от стенда.
Используются приёмники как в непосредственной близости к стенду, так и на значительном удалении (вплоть до 1000 км) от него.
В дополнение к двухчастотному кинематическому режиму PPP проанализирована полная ошибка позиционирования для обычного одночастотного режима по L1C коду.
Результаты показывают отсутствие видимых эффектов в точности позиционирования для обоих режимов во время работы стенда.
Итоги магистерской диссертации включены в следующий список публикаций:
\begin{enumerate}[leftmargin=*]
\item Yu. Yasyukevich, R. Vasilyev, K. Ratovsky, A. Setov, M. Globa, \begin{bf}S. Syrovatskii\end{bf}, A. Yasyukevich, A. Kiselev, A. Vesnin. Small-Scale Ionospheric Irregularities of Auroral Origin at Mid-Latitudes during the 22 June 2015 Magnetic Storm and Their Effect on GPS Positioning // \href{http://dx.doi.org/10.3390/rs12101579}{Remote Sensing}. --- 2020. --- Vol. 12, no. 10. 
\item Ю.В. Ясюкевич, \begin{bf}С.В. Сыроватский\end{bf}, А.М. Падохин, В.Л. Фролов, А.М. Веснин, Д.А. Затолокин, Г.А. Курбатов, Р.В. Загретдинов, А.В. Першин, А.С. Ясюкевич. Точность позиционирования GPS в различных режимах при активном воздействии на ионосферу мощным КВ-излучением нагревного стенда СУРА // Известия вузов. Радиофизика. --- 2020. 
\end{enumerate}

В заключение автор выражает благодарность за курирование в процессе подготовки дипломной работы научному руководителю -- к.ф.-м.н., с.н.с. кафедры физики атмосферы МГУ им. М.В. Ломоносова \begin{bf}Падохину Артёму Михайловичу\end{bf}. 
Также автор выражает глубокую благодарность за всестороннюю помощь научному консультанту -- к.ф.-м.н., в.н.с. ИСЗФ СО РАН \begin{bf}Ясюкевичу Юрию Владимировичу\end{bf}.  