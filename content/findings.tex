\chapter*{ВЫВОДЫ}
\addcontentsline{toc}{chapter}{ВЫВОДЫ} 

В целом, результаты, полученные в ГЛАВЕ 3, согласуются с результатами работы \cite{Luo2018}.
Во время геомагнитных бурь точность позиционирования PPP может снижаться в несколько раз.
При этом области снижения пространственно коррелируют с областями максимальных вариаций TEC.
В основном это авроральные регионы, в которых наиболее благоприятно образование неоднородностей электронной концентрации, оказывающих максимальное влияние на распространение радиосигналов, приводящих к их рассеянию, мерцаниям и т.п.
Однако стоит отметить, что зарегистрированные на средних широтах ухудшения точности позиционирования PPP превышают результаты работы \cite{Luo2018} (менее 0,32 м) и по своей величине сравнимы с ошибками, наблюдаемыми во время мощной солнечной радиовспышки класса X9.3 6 сентября 2018 года (более 0,5 м) \cite{Yasyukevich2018}, но по времени длятся дольше.
Более того, анализ показал, что для рассмотренных геомагнитных бурь 25-26 августа 2018 года и 21-23 июня 2015 года этот эффект был даже продолжительнее, чем во время более мощной бури в День святого Патрика (17 марта) 2015 года. 

Согласно результатам, полученным в ГЛАВЕ 4, можно сказать, что видимых эффектов в точности позиционирования (как в двухчастотном кинематическом режиме PPP, так и в обычном одночастотном режиме по L1C коду) во время работы нагревательного стенда СУРА обнаружено не было. 
Возможными причинами, объясняющими этот факт, являются относительно малая область нагрева и низкая амплитуда генерируемых ионосферных возмущений.
В связи с локализацией области нагрева только небольшая часть лучей приёмник-спутник может пройти через неё.
Это приводит к малому статистическому весу возможных ошибок при решении навигационной задачи и значительному снижению итоговой полной ошибки. 
Для приёмных станций, расположенных на значительном расстоянии от стенда, возможное влияние искусственных ПИВ оказывается замаскированным естественной изменчивостью среднеширотной ионосферы.
При этом вариации TEC с амплитудой 0,5 TECU приводят к вариациям дальности луча приёмник-спутник порядка 10 см. 
Заметить подобные эффекты относительно фоновых ошибок обычного одночастотного режима достаточно трудно.
Тем не менее даже в более перспективном (точном и чувствительном) режиме PPP теоретически предсказанные эффекты в точности позиционирования на практике при рассмотренных экспериментах подтвердить не удалось.