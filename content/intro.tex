\chapter*{ВВЕДЕНИЕ}
\addcontentsline{toc}{chapter}{ВВЕДЕНИЕ} 

За последние пару десятков лет Глобальные Навигационные Спутниковые Системы (ГНСС) нашли своё применение в различных сферах жизни человека.
Недавние маркетинговые исследования Европейского агентства по ГНСС \cite{EGA2017} показывают, что в период с 2015 по 2025 год мировой доход от услуг, использующих технологию ГНСС, увеличится примерно в 4 раза, а количество приёмных блоков ГНСС увеличится более чем в 2 раза и составит порядка 9 миллиардов единиц. 
Таким образом, технология ГНСС является актуальной и развивающейся.

В настоящее время существуют две полностью функционирующих ГНСС: GPS (США) с июля 1995 года и ГЛОНАСС (Россия) с октября 2011 года, а также две находятся на стадии разработки: Galileo (ЕС) и BeiDou (Китай).
Помимо этого, существуют региональные навигационные спутниковые системы, использование которых возможно только в определённых районах.
К ним относятся QZSS (Япония) и IRNSS (Индия).

По сравнению с другими системами позиционирования (наземными радионавигационными и инерциальными навигационными) ГНСС является наиболее точной.
Однако некоторые приложения ГНСС требуют повышенной точности и надёжности при определении координат.
К ним относятся точное земледелие, геодезия, беспилотные автомобили, авиация и т.д.  
Поэтому вопрос точности позиционирования ГНСС играет принципиальную роль.

Состояние ионосферы является одним из ключевых факторов, которые определяют точность позиционирования ГНСС. 
При распространении через ионосферу сигналы ГНСС подвергаются рассеянию на неоднородностях электронной концентрации. 
Наибольший эффект оказывают неоднородности размера порядка радиуса первой зоны Френеля $\sqrt{\lambda Z}$ ($\lambda$ -- длина волны сигнала, $Z$ -- расстояние до неоднородности). 
Для рабочих частот ГНСС размер таких неоднородностей варьируется от 100 до 300 м (так называемые мелкомасштабные неоднородности). 

Условно говоря, неоднородности могут быть результатом естественных и искусственных возмущений ионосферы.
Естественные возмущения ионосферы могут быть обусловлены внезапным повышением солнечной активности (например, солнечные вспышки), которое сопровождается магнитной бурей.  
С другой стороны существуют экспериментальные установки по изменению состояния ионосферы путем её нагрева с помощью мощных высокочастотных волн. 
Одной из таких экспериментальных установок является радиокомплекс СУРА, который расположен вблизи Нижнего Новгорода (56,15\degree N, 46,1\degree E).

В настоящее время существуют и используются два основных метода позиционирования.
Первый -- метод абсолютного позиционирования (absolute positioning), также известный как Single Point Positioning (SPP).
Этот метод позволяет одному приёмнику напрямую определять ``абсолютные'' координаты точки относительно системы координат WGS84.
Второй -- метод относительного позиционирования (relative positioning), который иногда называется методом дифференциального позиционирования (differencial positioning). 
Этот метод требует использования уже, как минимум, двух приёмников: координаты пользовательского приёмника определяются относительно известных координат опорных (сетевых) приёмников. 
Поскольку в методе дифференциального позиционирования используются измерения, выполняемые одновременно на нескольких приёмниках, то многие ошибки (ионосферная и тропосферная задержки, ошибки часов и координат спутника) могут быть устранены путем формирования разницы между измерениями. 
Поэтому метод дифференциального позиционирования широко используется для приложений, которые требуют высокой точности.
Однако эффективность этого метода в значительной степени зависит от расстояния между приёмниками: при увеличении расстояния качество позиционирования ухудшается.
Это является его основным ограничением. 
Кроме того, требование одновременной работы по меньшей мере двух приёмников технически реализуется более трудно и дорого, что делает метод дифференциального позиционирования менее привлекательным для большинства приложений.
В попытке преодолеть сетевую зависимость и удовлетворить высокую точность был разработан Precise Point Positioning (PPP) метод.
Первоначально PPP был ввёден для статических приложений \cite{Zumberge1997}, но в дальнейшем был изменён и для кинематических приложений.
PPP расширяет возможности SPP, используя точные эфемериды спутников, полученные, например, от IGS \cite{Dow2009}, вместо обычного широковещательного навигационного сообщения.

В соответствии с вышеизложенным, данная работа состоит из двух частей.
Первая часть посвящена анализу ошибки позиционирования во время нескольких сильных геомагнитных бурь за последнее десятилетие в глобальном масштабе.
Основная цель заключается в том, чтобы проанализировать изменения точности позиционирования в различных широтно-долготных областях.
Вторая часть работы посвящена анализу ошибки позиционирования во время работы нагревательного стенда СУРА.
Здесь анализируется возможность влияния мощного высокочастотного радиоизлучения на качество позиционирования GPS как вблизи области возмущения, так и на значительном расстоянии от неё.