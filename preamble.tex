\usepackage[T2A]{fontenc}
\usepackage[utf8]{inputenc}
\usepackage[english,russian]{babel}

% Поля
\usepackage[paper=a4paper, left=3cm, right=2cm, top=2cm, bottom=2cm]{geometry}

% Отступ первого абзаца
\usepackage{indentfirst}

% Названия глав, секций и подсекций
\usepackage{titlesec} 
\titleformat{\chapter}[display]{\centering}{\large\MakeUppercase{\chaptertitlename}\ \thechapter}{1em}{\large}
\titleformat{\section}{\centering}{\thesection}{0.5em}{}
\titleformat{\subsection}{\raggedright}{\thesubsection}{0.5em}{}
% {left spacing}{before spacing}{after spacing}
\titlespacing{\chapter}{0em}{-1em}{1em} 
\titlespacing{\section}{0em}{1em}{1em} 
\titlespacing{\subsection}{0em}{0.5em}{0.5em} 

% Ссылки
\usepackage[colorlinks=true, allcolors=blue, linktocpage=true]{hyperref}
\urlstyle{same}

% Литература и ссылки на литературу
\usepackage[numbers,compress]{natbib}
\renewcommand{\bibpreamble}{\vskip1em} % Отступ от заголовка литературы
\makeatletter
\renewcommand{\@biblabel}[1]{#1.}
\makeatother
\newcommand{\doi}[1]{\href{https://doi.org/#1}{doi: #1}}

% Оглавление
\usepackage[titles]{tocloft}
\usepackage{calc} % Ширина слова

\renewcommand{\cftchappresnum}{ГЛАВА~}
\makeatletter
\g@addto@macro\appendix{
  \addtocontents{toc}{
    \protect\renewcommand{\protect\cftchappresnum}{ПРИЛОЖЕНИЕ~}
    \addtolength{\cftchapnumwidth}{\widthof{ЖЕНИЕ~}}
  }
}
\makeatother

\renewcommand{\cftsecfont}{\small} % Размер шрифта секций в оглавлении

\addtolength{\cftchapnumwidth}{\widthof{ГЛАВА~}} % 
\setlength{\cftsecindent}{\widthof{ГЛАВА~}}      % Костыль для отступа глав и секций
\setlength{\cftsubsecindent}{\widthof{ГЛАВА~1~}} %

\setlength{\cftbeforechapskip}{0.5em} % Отступ между главами
\setlength{\cftbeforesecskip}{0.25em}  % Отступ между секциями

\renewcommand{\cftchapfont}{\normalfont}     % Главы не жирные
\renewcommand{\cftchappagefont}{\normalfont} % Номера страниц не жирные

\renewcommand{\cftchapleader}{\cftdotfill{\cftdotsep}} % Точки у глав

\renewcommand{\cfttoctitlefont}{\hfill\MakeUppercase} % Оглавление по середине
\renewcommand{\cftaftertoctitle}{\hfill\hfill}        %

\renewcommand{\bibsection}{} % Литература

% Нумерация картинок и таблиц
\usepackage{chngcntr}            
\counterwithout{figure}{chapter}
\counterwithout{table}{chapter}

% Списки
\usepackage{enumitem}

% Подписи
\usepackage{caption}   
\captionsetup[table]{justification=RaggedLeft,labelsep=newline}
\captionsetup[figure]{labelsep=period}

% Чтобы переносы слов были, как в ворде 
\usepackage[none]{hyphenat}
\sloppy

% Таблицы
\usepackage{multirow}
\usepackage[table,xcdraw]{xcolor}
\usepackage{longtable}

% Для альбомной ориентации
\usepackage{pdflscape}

% Misc
\usepackage{amsmath}
\usepackage{amssymb}
\usepackage{gensymb}
\usepackage{textcomp}
\usepackage{mathtools}