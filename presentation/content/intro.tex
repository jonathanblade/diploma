\begin{frame}{\textsc{Введение}}
\begin{tikzpicture}[
    start chain,
    block/.style={on chain,draw,rectangle,rounded corners,align=center,minimum width=110pt,scale=0.95},
    outblock/.style={draw,rectangle,dashed},
    >=latex
]
\node[block] (1) {Естественная \\ (гелиогеофизические \\ явления)};
\node[block,below=of 1] (2) {Искусственная \\ (нагревательные \\ радиокомплексы)};
\node[outblock,fit=(1) (2),label=Ионизация ионосферы] (3) {};
\node[block,right=of 1] (4) {Крупные};
\node[block,right=of 3] (5) {Средние};
\node[block,right=of 2] (6) {Мелкие \\ ($\sim\sqrt{\lambda Z}$)};
\node[outblock,fit=(4) (5) (6),label={[align=center]above:Неоднородности \\ электронной \\ концентрации}] (7) {};
\node[block,right=of 4] (8) {Рефракционные};
\node[block,right=of 6] (9) {Дифракционные};
\node[outblock,fit=(8) (9),label=Эффекты] (10) {};
\node[block,below=of 9] (11) {Точность \\ позиционирования \\ ГНСС};
\draw[->] (3) -- (7); 
\draw[->] (4) -- (8);
\draw[->] (5) -| (8); 
\draw[->] (6) -- (9); 
\draw[->] (8) -- ($(8.east)+(0.5,0)$) |- (11.east) node[pos=0.25,above,rotate=-90] {Ионосферная задержка};
\draw[->] (9) -- (11) node[midway,left] {Мерцания} ;
\end{tikzpicture}
\end{frame}