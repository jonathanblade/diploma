\section{\textsc{Метод исследования}}

\begin{frame}{\textsc{Метод исследования}}
\textbf{Precise Point Positioning (PPP)} – это метод абсолютного позиционирования, который использует кодовые или фазовые измерения сигнала в комбинации с точными эфемеридами спутников.

\vspace{1em}
В работе используется двухчастотная кинематическая модель PPP на основе недифференцированных и некомбинированных измерений GPS (псевдодальности и фазы несущей), которая реализуется при помощи программного обеспечения с открытым исходным кодом GAMP [Zhou et al., 2018].

\vspace{1em}
F. Zhou, D. Dong, W. Li, et al. GAMP: An open-source software of multi-GNSS precise point positioning using undifferenced and uncombined observations // \href{http://dx.doi.org/10.1007/s10291-018-0699-9}{GPS Solutions}. --- 2018. Vol. 22. 
\end{frame}

\begin{frame}{\textsc{Метод исследования}}{\textit{Модель PPP}}
\begin{equation*}
\begin{aligned}
R_r^s&=\underbrace{\vec{u}\cdot\vec{x}}_{\text{\shortstack{\\[0.85em] Геометрическое \\ расстояние}}}+\underbrace{c(\delta_r-\delta^s)}_{\text{\shortstack{\\[0.5em] Смещения \\ часов}}}+\underbrace{F\times T_V}_{\text{\shortstack{\\[0.5em] Тропосферная \\ задержка}}}+\underbrace{\alpha_fI_{L1}}_{\text{\shortstack{\\[0.5em] Ионосферная \\ задержка}}}+\underbrace{c(d_r-d^s)}_{\text{\shortstack{\\[0.5em] Некалиброванные \\ задержки}}}+\underbrace{e_r^s}_{\text{\shortstack{\\[0.5em] Другие \\ ошибки}}} \\
\lambda\Phi_r^s&=\overbrace{\vec{u}\cdot\vec{x}}^{\text{~~~~~~~~~~~~~~~~~~~~~~}}+\overbrace{c(\delta_r-\delta^s)}+\overbrace{F\times T_V}^{\text{~~~~~~~~~~~~~~~~~~~}}-\overbrace{\alpha_fI_{L1}}^{\text{~~~~~~~~~~~~~~~~~~~~}}+\overbrace{\lambda(N+b_r-b^s)}+\overbrace{e_r^s}^{\text{~~~~~~~~~}}
\end{aligned} 
\end{equation*}
\begin{description}[wide]
\item[$R_r^s$] -- псевдодальность по кодовым измерениям;
\item[$\lambda\Phi_r^s$] -- псевдодальность по фазовым измерениям;
\item[$F$] -- наклонный коэффициент; $\alpha_f=\frac{f_{L1}^2}{f^2}$;  
\item[$r$] -- индекс, обозначающий приёмник;
\item[$s$] -- индекс, обозначающий спутник.     
\end{description}
\end{frame}