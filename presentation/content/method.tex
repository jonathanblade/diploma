\begin{frame}{\textsc{Метод исследования}}
\textbf{Precise Point Positioning (PPP)} – это метод абсолютного позиционирования, который использует кодовые или фазовые измерения сигнала в комбинации с точными эфемеридами спутников.

\vspace{1em}
В работе используется двухчастотная кинематическая модель PPP на основе недифференцированных и некомбинированных измерений GPS (псевдодальности и фазы несущей), которая реализуется при помощи программного обеспечения с открытым исходным кодом GAMP [Zhou et al., 2018].

\vspace{1em}
F. Zhou, D. Dong, W. Li, et al. GAMP: An open-source software of multi-GNSS precise point positioning using undifferenced and uncombined observations // \href{http://dx.doi.org/10.1007/s10291-018-0699-9}{GPS Solutions}. --- 2018. Vol. 22. 
\end{frame}

\begin{frame}{\textsc{Метод исследования}}
\begin{equation*}
\begin{aligned}
R_r^s&=\underbrace{\vec{u}\cdot\vec{x}}_{\text{\shortstack{\\[0.85em] Геометрическое \\ расстояние}}}+\underbrace{c(\delta_r-\delta^s)}_{\text{\shortstack{\\[0.5em] Смещения \\ часов}}}+\underbrace{F\times T_V}_{\text{\shortstack{\\[0.5em] Тропосферная \\ задержка}}}+\underbrace{\alpha_fI_{L1}}_{\text{\shortstack{\\[0.5em] Ионосферная \\ задержка}}}+\underbrace{c(d_r-d^s)}_{\text{\shortstack{\\[0.5em] Некалиброванные \\ задержки}}}+\underbrace{e_r^s}_{\text{\shortstack{\\[0.5em] Другие \\ ошибки}}} \\
\lambda\Phi_r^s&=\overbrace{\vec{u}\cdot\vec{x}}^{\text{~~~~~~~~~~~~~~~~~~~~~~}}+\overbrace{c(\delta_r-\delta^s)}+\overbrace{F\times T_V}^{\text{~~~~~~~~~~~~~~~~~~~}}-\overbrace{\alpha_fI_{L1}}^{\text{~~~~~~~~~~~~~~~~~~~}}+\overbrace{\lambda(N+b_r-b^s)}+\overbrace{e_r^s}^{\text{~~~~~~~~~}}
\end{aligned} 
\end{equation*}
\begin{tikzpicture}[
    start chain,
    block/.style={on chain,draw,rectangle,rounded corners,align=center,minimum width=60pt,scale=0.95},
    outblock/.style={draw,rectangle,dashed},
    >=latex,
    node distance=0.5cm
]
\node[block] (1) {obs \\ (файлы наблюдений)};
\node[block,below=of 1] (2) {sp3 \\ (точные эфемериды)};
\node[on chain,below=of 2] (3) {...};
\node[outblock,fit=(1) (2) (3),label=Входные данные] (4) {};
\node[block,right=0.8cm of 1] (5) {Модель \\ PPP};
\node[block,right=0.8cm of 2] (6) {Алгоритм \\ PPP};
\node[outblock,fit=(5) (6),label=GAMP] (7) {};
\node[block,right=of 7,label={[align=center]above: Истинные \\ координаты}] (8) {
    \begin{math}
    \begin{aligned}
    x_0=\text{median}(x) \\
    y_0=\text{median}(y) \\
    z_0=\text{median}(z)
    \end{aligned}
    \end{math}};
\node[block,right=of 8,label={[align=center]above: Ошибка \\ позиционирования}] (9) {
    \begin{math}
    \sigma_{xyz}=\sqrt{\begin{aligned}&(x-x_0)^2+\\&(y-y_0)^2+\\&(z-z_0)^2\end{aligned}}
    \end{math}};
\draw[->] (4) -- (7);
\draw[->] (7) -- (8);
\draw[->] (8) -- (9);   
\end{tikzpicture}
\end{frame}