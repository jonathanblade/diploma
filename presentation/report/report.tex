\documentclass[a4paper,14pt]{extarticle}

\title{\textsc{Оценка точности позиционирования GPS при естественных и искусственных возмущениях ионосферы}}
\author{\textbf{Сыроватский Семён Владиславович} \\ {\small Научный руководитель: к.ф.-м.н., с.н.с. Падохин Артём Михайлович} \\[1em] Кафедра физики атмосферы МГУ им. М.В. Ломоносова}
\date{2 июня 2020 г.}

\usepackage[T2A]{fontenc}
\usepackage[utf8]{inputenc}
\usepackage[english,russian]{babel}

% Поля
\usepackage[paper=a4paper, left=3cm, right=2cm, top=2cm, bottom=2cm]{geometry}

% Подписи
\usepackage{amsmath}

% Знак градуса
\usepackage{textcomp}
\usepackage{gensymb}

% Отступ первого абзаца
\usepackage{indentfirst}

% Названия глав, секций и подсекций
\usepackage{titlesec} 
\titleformat{\chapter}[display]{\centering}{\large\MakeUppercase{\chaptertitlename}\ \thechapter}{1em}{\large}
\titleformat{\section}{\centering}{\thesection}{0.5em}{}
\titleformat{\subsection}{\centering}{\thesubsection}{0.5em}{}
\titlespacing{\chapter}{0em}{-1em}{1em} % {left spacing}{before spacing}{after spacing}

% Ссылки
\usepackage[colorlinks=true, allcolors=blue, linktocpage=true]{hyperref}

% Литература и ссылки
\usepackage[numbers]{natbib}
\makeatletter
\renewcommand{\@biblabel}[1]{#1.}
\makeatother
% \newcommand{\doi}[1]{\href{http://doi.org/#1}{doi: #1}}

% Оглавление
\usepackage{tocloft}
\usepackage{calc} % ширина слова
% \setlength{\cftchapindent}{0em} 
\setlength{\cftbeforetoctitleskip}{-1em} % Отступ оглавления сверху
\setlength{\cftaftertoctitleskip}{1em}   % Отступ оглавления снизу

\renewcommand{\cftchappresnum}{\MakeUppercase{\chaptername~}}

\addtolength{\cftchapnumwidth}{\widthof{ГЛАВА~}} % 
\setlength{\cftsecindent}{\widthof{ГЛАВА~}}      % Костыль для отступа глав и секций
\setlength{\cftsubsecindent}{\widthof{ГЛАВА~1~}} %

\setlength{\cftbeforesecskip}{0.5em}    % Отступ между главой и секцией
\setlength{\cftbeforesubsecskip}{0.5em} % Отступ между секцией и подсекцией 

\renewcommand{\cftchapfont}{\normalfont}     % Главы не жирные
\renewcommand{\cftchappagefont}{\normalfont} % Номера страниц не жирные

\renewcommand{\cftchapleader}{\cftdotfill{\cftdotsep}} % Точки у глав

\renewcommand{\cfttoctitlefont}{\hfill\MakeUppercase} % Оглавление по середине
\renewcommand{\cftaftertoctitle}{\hfill\hfill}        %

\renewcommand{\bibsection}{\centering ЛИТЕРАТУРА} % Литература

% Картинки
\usepackage{graphicx}
\usepackage{chngcntr}            % Изменение нумерации
\counterwithout{figure}{chapter} % Нумерация картинок
\counterwithout{table}{chapter}  % Нумерация таблиц

\begin{document}
\maketitle
\section*{Слайд 2}
Состояние ионосферы является одним из ключевых факторов, которые определяют точность позиционирования ГНСС (глобальных навигационных спутниковых систем).
При распространении через ионосферу электромагнитные волны (в частности сигналы ГНСС) подвергаются взаимодействию с неоднородностями электронной концентрации, которые, условно говоря, могут быть результатом естественных и искусственных (техногенных) возмущений ионосферы.
Эти неоднородности существуют в широком спектральном диапазоне масштабов.
Средние и крупные неоднородности в основном обуславливают рефракционные эффекты и непосредственно влияют на ионосферную задержку сигнала, что может повлечь за собой ошибку позиционирования.
Мелкомасштабные неоднородности, размера порядка первой зоны Френеля $\sqrt{\lambda Z}$ ($\lambda$ -- длина волны сигнала, $Z$ -- расстояние до неоднородности), в основном обуславливают дифракционные эффекты.
С ними связаны мерцания сигнала (случайные флуктуации амплитуды и фазы сигнала), которые значительным образом могут ухудшить отношение сигнал/шум или вообще привести к срыву сигнала, что также может повлечь за собой ошибку позиционирования (изменяется геометрия навигационной задачи, т.е. изменяется конфигурация используемого спутникового созвездия).

\section*{Слайд 3}
В соответствии с этим, данная работа состоит из двух основных частей. 
Первая часть посвящена исследованию точности позиционирования GPS во время геомагнитных бурь.
Здесь необходимо исследовать влияние геомагнитных бурь в период 24-го цикла солнечной активности на качество позиционирования GPS в глобальном масштабе (в различных широтно-долготных областях).
Вторая часть посвящена исследованию точности позиционирования GPS во время работы нагревательного стенда СУРА.
Здесь необходимо исследовать возможность влияния ВЧ (высокочастотного) радиоизлучения антропогенного характера на качество позиционирования GPS как вблизи области возмущения, так и на значительном расстоянии от неё.

\section*{Слайд 4}
Чтобы заметить самые слабые эффекты, в работе используется прецизионный метод позиционирования PPP (Precise Point Positioning).
PPP -- это метод абсолютного позиционирования (т.е. приёмник напрямую определяет свои ``абсолютные'' координаты относительно системы WGS84), который использует кодовые или фазовые измерения сигнала в комбинации с точными эфемеридами\footnote{Эфемериды -- орбитальные данные (элементы Кеплера и параметры возмущения), используемые для предсказания траектории спутника.} спутников, предоставляемыми службами ГНСС (например, IGS).
Для частичного устранения влияния ионосферы применяется двухчастотная кинематическая модель PPP на основе недифференцированных и некомбинированных измерений GPS (псевдодальности и фазы несущей), которая реализуется при помощи программного обеспечения с открытым исходным кодом GAMP\footnote{F. Zhou, D. Dong, W. Li, et al. GAMP: An open-source software of multi-GNSS precise point positioning using undifferenced and uncombined observations // \href{http://dx.doi.org/10.1007/s10291-018-0699-9}{GPS Solutions}. --- 2018. Vol. 22.}.

\section*{Слайд 5}
На слайде представлен вид линеризованных уравнений псевдодальности по кодовым ($R_r^s$) и фазовым ($\lambda \Phi_r^s$) измерениям GPS (индекс $r$ соответствует приёмнику, а $s$ -- спутнику). 
Таким образом, в решении учитывается смещение часов приёмника $\delta_r$ и спутника $\delta^s$, вертикальная влажная тропосферная задержка $T_V$ ($F$ -- наклонный коэффициент), ионосферная задержка $\alpha_fI_{L1}$ ($\alpha_f=\frac{f_{L1}^2}{f^2}$), а также некалиброванные кодовые ($d$) и фазовые ($b$) задержки приёмника и спутника.
Помимо этого, моделируются менее значимые эффекты (смещение и вариации фазовых центров антенн приёмника и спутника, релятивистские эффекты, эффект Саньяка, гидростатическая задержка, приливные эффекты и wind-up эффект).
На вход GAMP подаются файлы obs (спутниковые наблюдения) и sp3 (точные эфемериды).
На выходе получается решение PPP за 24 часа измерений.
На его основе (без учёта времени сходимости) определяются истинные (медианные) геоцентрические координаты станции.
Полная ошибка позиционирования считается, как разница между текущей ($x$, $y$, $z$) и истинной ($x_0$, $y_0$, $z_0$) позицией:
\begin{equation*}
\sigma_{xyz}=\sqrt{(x-x_0)^2+(y-y_0)^2+(z-z_0)^2}    
\end{equation*}  

\section*{Слайд 6}
В качестве примера сначала рассмотрим геомагнитную бурю 25-26 августа 2018 года.
На панелях (a, b) изображены глобальные распределения вариаций TEC\footnote{Yu.V. Yasyukevich, A.V. Kiselev, I.V. Zhivetiev, et al. SIMuRG: System for Ionosphere Monitoring and Research from GNSS // \href{http://dx.doi.org/10.1007/s10291-020-00983-2}{GPS Solutions}. --- 2020. --- Vol. 24.} (полного электронного содержания), отфильтрованных в диапазоне периодов 10-20 мин, в 07:45 UT для 25 (до бури) и 26 (главная фаза бури) августа 2018 года, соответственно.
Панелям (c, d) соответствуют глобальные распределения полной ошибки позиционирования.  
На панели (b) во время главной фазы бури заметно значительное увеличение вариаций TEC в областях, вытянутых вдоль аврорального овала, как в северном, так и в южном полушариях.
В этих же регионах на панели (d) заметно увеличение полной ошибки позиционирования, которые не были зарегистрированы в период до бури.
Для некоторых приёмных станций величина ошибки превосходит 0,5 м, по сравнению с 0,1 м в спокойный день.

\section*{Слайд 7}
Чтобы оценить динамику ошибки позиционирования для различных широтно-долготных регионов, рассчитываются среднезональные значения (т.е. ошибка усредняется по ячейкам 2,5\degree по широте и долготе) для двух регионов: восточного ($0\div180\degree$E) и западного ($0\div180\degree$W) полушарий.
На слайде изображены зависимости усреднённой полной ошибки позиционирования от времени и широты для восточного (a, c) и западного (b, d) полушарий.
Панелям (a, b) соответствует 25 августа, а (с, d) -- 26 августа.
Горизонтальные синие полосы (нулевые значения) указывают на отсутствие данных.
Следует обратить внимание, что решение PPP, полученное при помощи GAMP, имеет период схождения примерно от 0 до 2 UT.
Поэтому это время можно не рассматривать.
В течение всего дня 26 августа в североамериканском секторе отчётливо наблюдается резкое снижение точности PPP.
Помимо этого, заметный рост ошибки виден на высоких широтах в южном полушарии до примерно 14 UT.
В восточном полушарии ухудшение точности позиционирования наблюдается до полудня на широтах 60-70\degree N.

\section*{Слайд 8}
Следующим рассмотренным гелиогеофизическим событием является магнитная буря 21-22 июня 2015 года.
На панелях (a, b) изображены глобальные распределения вариаций TEC в 19:00 UT для 21 и 22 июня 2015 года, соответственно.
Панелям (c, d) соответствуют глобальные распределения полной ошибки позиционирования.
Аналогично предыдущему случаю здесь значительное увеличение вариаций TEC наблюдается в областях, вытянутых вдоль аврорального овала, как в северном, так и в южном полушариях. 
Однако можно заметить, что “интенсивность” и долготная протяжённость этих областей гораздо больше, чем для бури 25-26 августа 2018 года. 
Абсолютные значения вариаций TEC достигают более 0,4 TECU вплоть до азиатского региона.
Эти области также хорошо пространственно коррелируют с областями увеличения полной ошибки позиционирования на панели (d).

\section*{Слайд 9}
На слайде изображены зависимости усреднённой ошибки позиционирования от времени и широты для восточного (a, c) и западного (b, d) полушарий.
Панелям (a, b) соответствует 21 июня, а (c, d) -- 22 июня.
Вертикальной пунктирной линией отмечено время начала внезапной магнитной бури в 18:30 UT.
Аналогично, 22 июня отчётливо видно резкое ухудшение точности позиционирования после 18:30 UT в северном полушарии на широтах 65-75\degree N, а также в южном полушарии на широтах 75-90\degree S.
Здесь величина ошибки увеличивается в несколько раз и превышает 0,5 м.
Помимо этого, на панели (d) заметно экваториальное ``распространение'' ошибки со временем вплоть до 50\degree N.

\section*{Слайд 10}
Подводя итоги первой части работы, можно сделать следующие выводы:
\begin{enumerate}[leftmargin=*]
\item Рассмотрены две геомагнитные в период 24-го цикла солнечной активности: за 25-26 августа 2018 года и 21-22 июня 2015 года.
\item Установлено, что точность позиционирования PPP на средних и высоких широтах может снижаться в несколько (до 5) раз.
\item Области снижения пространственно коррелируют с областями максимальных вариаций TEC (авроральные регионы).
\item Зарегистрированные эффекты по величине сравнимы с эффектами от мощной солнечной вспышки класса X9.3 6 сентября 2018 года и по времени продолжительнее, чем эффекты во время более мощной бури в День святого Патрика (17 марта) 2015 года.
\end{enumerate}

С более подробными результатами можно ознакомится в самой магистерской диссертации, а также в статье Yu. Yasyukevich, R. Vasilyev, K. Ratovsky, A. Setov, M. Globa, \begin{bf}S. Syrovatskii\end{bf}, A. Yasyukevich, A. Kiselev, A. Vesnin. Small-Scale Ionospheric Irregularities of Auroral Origin at Mid-Latitudes during the 22 June 2015 Magnetic Storm and Their Effect on GPS Positioning // \href{http://dx.doi.org/10.3390/rs12101579}{Remote Sensing}. --- 2020. --- Vol. 12, no. 10. 

\section*{Слайд 11}
Вторая часть работы посвящена исследованию точности позиционирования GPS во время работы нагревательного стенда СУРА.
СУРА -- экспериментальный радиокомплекс, который был создан в 1981 году для проведения прикладных и фундаментальных исследований состояния ионосферы при воздействии на неё ВЧ излучением техногенного характера.
\href{https://clck.ru/NfWnc}{Расположен} вблизи Нижнего Новгорода (56,15\degree N, 46,1\degree E).
Основу стенда составляют три ВЧ радиопередатчика, каждый из которых имеет мощность 250 кВт и работает в диапазоне частот 4-25 МГц, а также антенная решетка размером примерно $300\times300\,\text{м}^2$, которая позволяет излучать волны O- и X- поляризации в диапазоне частот 4,3-9,5 МГц.
Эффективная мощность излучения (в случае синхронной работы трёх модулей) и ширина диаграммы направленности составляют 80-240 МВт и 6-12\degree, соответственно.
В работе используются двухчастотные измерения GPS 14 приёмных станций, изображённых на слайде.
Наиболее близкие приёмники были размещены на самом полигоне (SURA и PREG), а также на удалении до 20 км от него (ZASU и VORO).
Остальные 10 приёмников находились на расстоянии от 23 до 1217 км от стенда.

\section*{Слайд 12}
На слайде изображены динамики полной ошибки позиционирования в двухчастотном кинематическом режиме PPP для 23 августа 2010 года.
Серые вертикальные полосы означают сеансы работы стенда.
Станции упорядочены по мере удаления от стенда сверху (самая ближняя) вниз (самая удалённая).
Помимо этого, для приёмника VORO добавлена динамика индекса AATR (стандартного отклонения производной TEC, нормированной на квадрат наклонного коэффициента) и вариаций TEC (dTEC) с окном фильтрации 10-20 мин, которое было выбрано на основе характерных периодов работы стенда. 
Обрыв данных для станции ZASU обусловлен отключением электричества.
Максимальное ухудшение точности позиционирования ожидалось именно для этой станции в силу её удаления от стенда и расположения антенны (видимая область -- область нагрева).
В целом видно, что уровень ошибки для ZASU выше, чем для других приёмников.
Однако её динамика, как и для других станций, не коррелирует со временем работы стенда, а максимальные значения, наоборот, соответствуют паузе в режиме работы.
Аналогичные результаты (но с другой доступной комбинацией приёмников) также наблюдаются для остальных двух экспериментальных дней (19 и 20 сентября 2016 года).
В дополнение к двухчастотному кинематическому режиму PPP была рассчитана полная ошибка позиционирования для обычного одночастотного режима по L1C коду.
В данном режиме реальное состояние ионосферы не учитывается, поэтому ухудшение точности позиционирования из-за активного воздействия более вероятно.
Тем не менее зафиксировать синхронные увеличения ошибки, связанные с работой стенда СУРА, также не получилось.

\section*{Слайд 13}
На слайде изображена таблица, резюмирующая среднее и среднее и среднеквадратическое отклонение (СКО) полной ошибки позиционирования во время сеансов и пауз работы стенда.
Усреднения производятся за весь день.
Интересно, но результаты также показывают более низкие значения ошибки во время нагрева, при этом СКО во время пауз, наоборот, практически всегда больше.
Возможно, это обусловлено разной статистикой.

\section*{Слайд 14}
Подводя итоги второй части работы, можно сделать следующие выводы:
\begin{enumerate}[leftmargin=*]
\item Рассмотрены три экспериментальных дня работы нагревательного стенда СУРА: за 23 августа 2010 года и 19-20 сентября 2016 года.
\item Эффектов ухудшения точности позиционирования как в режиме PPP, так и в обычном одночастотном режиме не обнаружено.
\end{enumerate}

Возможными причинами, объясняющими такой результат, являются относительно малая область возмущений и низкая амплитуда генерируемых ионосферных возмущений.
В связи с локализацией области нагрева только небольшая часть лучей приёмник-спутник может пройти через неё.
Это приводит к малому статистическому весу возможных ошибок при решении навигационной задачи и значительному снижению итоговой полной ошибки.
Для приёмных же станций, расположенных на значительном расстоянии от стенда, возможное влияния искусственных ПИВ (перемещающихся ионосферных возмущений) может быть замаскированно естественной изменчивостью среднеширотной ионосферы. 
С более подробными результатами также можно ознакомится в самой магистерской диссертации, а также в статье Ю.В. Ясюкевич, \begin{bf}С.В. Сыроватский\end{bf}, А.М. Падохин, В.Л. Фролов, А.М. Веснин, Д.А. Затолокин, Г.А. Курбатов, Р.В. Загретдинов, А.В. Першин, А.С. Ясюкевич. Точность позиционирования GPS в различных режимах при активном воздействии на ионосферу мощным КВ-излучением нагревного стенда СУРА // Известия вузов. Радиофизика. --- 2020.
\end{document}